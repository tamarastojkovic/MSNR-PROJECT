\documentclass[compress, containsverbatim,mathserif, xcolor=dvipsnames, unicode]{beamer}


%tema
\usetheme{Antibes}
\usecolortheme[named=Maroon]{structure}
\usecolortheme{beaver}
\setbeamercolor{titlelike}{fg=black, bg=purple}
\setbeamercolor{palette primary}{fg=pink, bg=purple}
\setbeamercolor{palette secondary}{fg=purple, bg=pink}
\setbeamercolor{palette tertiary}{bg=purple, fg=black}
\setbeamercolor{palette quaternary}{bg=purple, fg=black}
\setbeamercolor{frametitle}{bg=purple}
\setbeamercolor{block body}{bg=pink, fg=black}
\setbeamercolor{block title}{bg=purple, fg=white}
\setbeamertemplate{itemize item}{\color{purple}$\blacksquare$}
\setbeamertemplate{itemize subitem}{\color{purple}$\blacksquare$}
\setbeamercolor{item projected}{bg=purple,fg=white}
\setbeamercolor{section in toc}{fg=purple,bg=white}





% zbog srpskog
\usepackage[serbian]{babel}
\usepackage[utf8]{inputenc}
\usepackage[T2A]{fontenc}

% za matematiku
\usepackage{amsmath}
\usepackage{amssymb}
\usepackage{fontawesome5}

\usepackage{xcolor}

\newtheorem{primer}{Primer}[section]

\title{Optimizacije kroz GCC, LLVM i Native Image}
\author{Tamara Stojković, Emilija Stošić, Teodora Isailović}
\institute{Metodologija stručnog i naučnog rada \\ Matematički fakultet \\ Univerzitet u Beogradu}
\date{
	\footnotesize{Beograd, decembar 2022.}	
}

\begin{document}

\begin{frame}
\titlepage

\begin{figure}[h!]
    \centering
    \begin{flushleft}
    \includegraphics[width=15mm]{logo.png}
    \end{flushleft}
\end{figure}
\end{frame}

\begin{frame}{Sadržaj}
\tableofcontents
\end{frame}

\section{Uvod}
\begin{frame}{Uvod}
\vspace{\baselineskip}
\begin{itemize}
	\uncover<1->{\item Optimizacija - tehnika transformacije dela programa, sa ciljem poboljšanja performansi koda} 
	\uncover<2->{\item Oblast u kojoj se danas vrši većina istraživanja kompajlera}
    \uncover<3->{\item Kompajleri - različiti nivoi optimizacije}
	\uncover<4->{\item Na nivoima kompromisi između mera - kvalitet koda, veličina koda, vreme kompilacije}
    \uncover<5->{\item Izbor kompajlera i nivoa optimizacije zavise od konkretnog programa}
\end{itemize}
\end{frame}



\section{Osnovna podela optimizacija}
\subsection{Optimizacije međukoda}
\begin{frame}{Lokalne optimizacije}
    
    \begin{itemize}
        \uncover<1->{\item Razlikujemo: lokalne, globalne i međuproceduralne}
        \uncover<2->{\item  \textcolor{purple}{Lokalne optimizacije} služe za ubrzavanje malih delova neke funkcije, najlakše za izvođenje}
        \uncover<3->{\item Tehnike lokalne optimizacije koje razlikujemo: }
                \begin{itemize}
                    \uncover<4-> {\item Eliminacija čestih podizraza}
                    \uncover<5-> {\item Slaganje konstanti}
                    \uncover<6-> {\item Propagacija kopija}
                    \uncover<7-> {\item Smanjenje snage operatora}
                    \uncover<8-> {\item Eliminacija mrtvog koda}
                    \uncover<9-> {\item Algebarsko pojednostavljenje i reasocijacija}
                    \uncover<10-> {\item Kompozicija lokalnih transformacija}
                \end{itemize}
        \end{itemize}
\end{frame}


\begin{frame}{Globalne i međuproceduralne optimizacije}
    \begin{itemize}
        \uncover<1->{\item \textcolor{purple}{Globalne optimizacije} primenjuju se na jednu po jednu funkciju, slične lokalnim}
        \uncover<2->{\item Globalne optimizacije koje razlikujemo: }
                \begin{itemize}
                    \uncover<3-> {\item Globalna eliminacija mrtvog koda}
                    \uncover<4-> {\item Globalno propagiranje konstanti}
                    \uncover<5-> {\item Globalna eliminacija čestih podizraza}
                    \uncover<6-> {\item Optimizacija kretanje koda}
                    \uncover<7-> {\item Pomeranje invarijantog koda}
                    \uncover<8-> {\item Parcijalna eliminacija suvišnosti}
                \end{itemize}
        \uncover<9->{\item \textcolor{purple}{Međuproceduralna optimizacija } - radi na celokupnom grafu kontrole toka} 
        \uncover<10->{\item Vrši se na nivou celog programa tj. više funkcija }
        \uncover<11->{\item Najpoznatija tehnika je uvlačenje definicija funkcija }
        \end{itemize}
\end{frame}


\subsection{Optimizacije koda}
\begin{frame}{Optimizacije koda}
    
    \begin{itemize}
        \uncover<1->{\item Optimizovani međukod se prevodi u asemblerski tj. mašinski kod - faza generisanja}
        \uncover<2->{\item  Bitne su specifične karakteristike mašine}
        \uncover<3->{\item Razlikujemo  sledeće tehnike: }
                \begin{itemize}
                    \uncover<4-> {\item Optimizacija redosleda instrukcija - obuhvata fazu odabira instrukcija, alokacije registara i raspoređivanja instrukcija}
                    \uncover<5-> {\item Optimizacija upotrebom keša - zasniva se na prostornoj i vremenskoj lokalnosti, cilj da bude što bolja}
                \end{itemize}
        \end{itemize}
\end{frame}







%samo je testirano dodavanje slika -ubacite koju hocete i kako hocete i koliko po slajdu
%\section{dodati naziv}
\begin{frame}{GCC optimizacija} %promeniti sve
    \begin{figure}[h!]
        \begin{center}
       \includegraphics[scale = 0.2]{../pics/test.png}
       \end{center}
       \caption{Rezultati optimizacije kod GCC kompajlera}
    \end{figure} 
\end{frame}


\end{document}

    
