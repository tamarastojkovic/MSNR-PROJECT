% !TEX encoding = UTF-8 Unicode
\documentclass[a4paper]{article}

\usepackage{color}
\usepackage{url}
\usepackage[T2A]{fontenc} % enable Cyrillic fonts
\usepackage[utf8]{inputenc} % make weird characters work
\usepackage{graphicx}

\usepackage[english,serbian]{babel}
%\usepackage[english,serbianc]{babel} %ukljuciti babel sa ovim opcijama, umesto gornjim, ukoliko se koristi cirilica

\usepackage[unicode]{hyperref}
\hypersetup{colorlinks,citecolor=green,filecolor=green,linkcolor=blue,urlcolor=blue}

\usepackage{listings}

%\newtheorem{primer}{Пример}[section] %ćirilični primer
\newtheorem{primer}{Primer}[section]

\definecolor{mygreen}{rgb}{0,0.6,0}
\definecolor{mygray}{rgb}{0.5,0.5,0.5}
\definecolor{mymauve}{rgb}{0.58,0,0.82}

\lstset{ 
  backgroundcolor=\color{white},   % choose the background color; you must add \usepackage{color} or \usepackage{xcolor}; should come as last argument
  basicstyle=\scriptsize\ttfamily,        % the size of the fonts that are used for the code
  breakatwhitespace=false,         % sets if automatic breaks should only happen at whitespace
  breaklines=true,                 % sets automatic line breaking
  captionpos=b,                    % sets the caption-position to bottom
  commentstyle=\color{mygreen},    % comment style
  deletekeywords={...},            % if you want to delete keywords from the given language
  escapeinside={\%*}{*)},          % if you want to add LaTeX within your code
  extendedchars=true,              % lets you use non-ASCII characters; for 8-bits encodings only, does not work with UTF-8
  firstnumber=1000,                % start line enumeration with line 1000
  frame=single,	                   % adds a frame around the code
  keepspaces=true,                 % keeps spaces in text, useful for keeping indentation of code (possibly needs columns=flexible)
  keywordstyle=\color{blue},       % keyword style
  language=Python,                 % the language of the code
  morekeywords={*,...},            % if you want to add more keywords to the set
  numbers=left,                    % where to put the line-numbers; possible values are (none, left, right)
  numbersep=5pt,                   % how far the line-numbers are from the code
  numberstyle=\tiny\color{mygray}, % the style that is used for the line-numbers
  rulecolor=\color{black},         % if not set, the frame-color may be changed on line-breaks within not-black text (e.g. comments (green here))
  showspaces=false,                % show spaces everywhere adding particular underscores; it overrides 'showstringspaces'
  showstringspaces=false,          % underline spaces within strings only
  showtabs=false,                  % show tabs within strings adding particular underscores
  stepnumber=2,                    % the step between two line-numbers. If it's 1, each line will be numbered
  stringstyle=\color{mymauve},     % string literal style
  tabsize=2,	                   % sets default tabsize to 2 spaces
  title=\lstname                   % show the filename of files included with \lstinputlisting; also try caption instead of title
}

\begin{document}

\title{Optimizacije u okviru kompajlera GCC/LLVM\\ \small{Seminarski rad u okviru kursa\\Metodologija stručnog i naučnog rada\\ Matematički fakultet}}

\author{Tamara Stojković, Emilija Stošić, Teodora Isailović \and tamara.stojkovic.1998@gmail.com, emilijazstosic@gmail.com, \and teodora.isailovic@gmail.com}
\date{12.~novembar 2022.}

\maketitle

\abstract{
Ovo se piše na kraju. }
\tableofcontents

\newpage

\section{Uvod}
\label{sec:uvod}
Optimizacija predstavlja tehniku transformacije dela programa,  tako da kod bude što je moguće efikasniji. 
Za cilj ima poboljšanje performansi koda, a ne savršen rezultat ( uglavnom se ne može se reći da predstavlja pronalazak „optimalnog rešenja“ ). 
Optimizacija je oblast u kojoj se danas vrši većina istraživanja kompajlera. 
U okviru savremenih kompajlera dostupan je veliki broj optimizacija koje sve međusobno deluju na razne načine i imaju uticaj na kvalitet koda, veličinu koda, vreme kompilacije, potrošnju energije itd. Kompajleri obično obezbeđuju ograničen broj  standardnih optimizacionih nivoa, kao što su O1, -O2,-O3 i -Os. Na ovim nivoima ostvaruju se kompromisi između  različitih mera kao što su kvalitet koda, veličina koda i vreme kompilacije. 
U ovom radu ćemo upravo predstaviti vrste optimizacija, sa akcentom na one koje su dostupne u okviru kompajlera GCC i LLVM.

\section{Osnovna podela optimizacija}
Postoje razne tehnike u primeni procesa optimizacije. Razlikujemo optimizacije koje se primenjuju na međukod,  kao deo generisanja ciljnog koda, ali i nakon generisanja ciljnog koda.
\textbf{Optimizacija međukoda} primenjuje se da se kod pojednostavi, preuredi ili sažme. Predstavlja optimizaciju koja ne uzima u obzir specifičnost ciljne arhitekture.
Neke od stavki koje \textbf{optimizacija kao deo generisanja ciljnog koda} omogućava su biranje instrukcija i određivanje načina alociranja objekata. Na kraju, može se izvršiti i \textbf{optimizacija ciljnog koda}, gde se pokušava prerada samog asemblerskog koda u nešto efikasnije. 
U ovom slučaju zahteva se detaljno poznavanje ciljne arhitekture, kao i asemblerskog i mašinskog jezika ciljnog programa.



\subsection{Optimizacije međukoda}
\label{subsec:optimizacija_međukod}
U okviru optimizacije međukoda razlikujemo :\textbf{lokalne}, \textbf{globalne} i \textbf{međuproceduralne optimizacije}.

\subsubsection{Lokalne optimizacije}
\label{subsubsec:lokalne}
Lokalne optimizacije služe za ubrzavanje malih delova neke funkcije i rade sa konketnim naredbama unutar osnovnog bloka(eng. basic block)  
One su obično najlakše za izvođenje jer nije neophodno raditi analizu kontrole toka(eng. ~{\em  Control - flow}).  
U nastavku navodimo neke od tehnika lokalne optimizacije koje se vrše nad međukodom.

\begin{itemize}
  \item \textbf{Eliminacija čestih podizraza (eng. ~{\em Common subexpression elimination} )} \newline
  Podrazumeva izbegavanje izvračunavanja čestog izraza više puta(ako se javlja više od jednom). 
  U slučaju da dve operacije daju isti rezultat, kažemo da su česte, pa je bolje izračunati izraz jednom i pozvati ga naredni put na mestu gde je neophodno.

  \item \textbf{Slaganje konstanti (eng. ~{\em Constant folding})} \newline
  Podrazumeva da se konstantni izrazi mogu evaluirati u vreme kompilacije. To uključuje utvrđivanje da svi operandi u izrazu imaju konstantnu vrednost, a zatim menjanje izraza njegovom vrednošću.
  Često je i lepše za pisanje i čitanje. 
  \item \textbf{Propagacija kopija (eng. ~{\em Copy propagation})} \newline
  Podrazumeva izbegavanje uvođenja promenljivih, koje samo čuvaju vrednosti nekih promenljivih koje već postoje.
  Posebno je značajna jer je u stanju da eliminiše veliki broj instrukcija, koje služe samo za kopiranje vrednosti iz jedne promenljive u drugu.
  \item \textbf{Smanjenje snage operatora (eng. ~{\em Operator strength reduction})} \newline
  Podrazumeva zamenu operatora „jeftinijim operatorom“. 
  Od same arhitekture zavisi koje operacije se smatraju najjeftinijim, a koje najskupljim, tako da je za dobru optimizaciju neophodno poznavati ciljnu mašinu.
  \item \textbf{Eliminacija mrtvog koda (eng. ~{\em Dead code elimination})} \newline
  Podrazumeva da ako se rezultat neke instrukcije nadalje ne koristi, instrukcija se smatra „mrtvom“ i može biti uklonjena. 
  Ovo omogućava uprošćavanje koda, izbacivanjem nepotrebnih izračunavanja.
  \item \textbf{Algebarsko pojednostavljenje i reasocijacija (eng. ~{\em Algebaric simplification and reassociation})} \newline
  Podrazumeva pojednostavljenje izraza korišćenjem algebarskih svojstava(zakona algebre). Reasocijacija podrazumeva korišćenje svojstava kao što su asocijativnost, komutativnost i distributivnost kako bi se nakon toga omogućila lakša primena drugih optimizacija.
  \item \textbf{Kompozicija lokalnih transformacija} \newline
  Različite optimizacije koje smo do sada videli, brinu o malom delu koda. 
  Za maksimalan efekat, možda će morati neke optimizacije da se primene više puta. 

\end{itemize}


\subsubsection{Globalne optimizacije}
\label{subsubsec:globalne}
Predstavljaju optimizacije koje su slične lokalnim i koje možemo primeniti sa nekim dodatnim analizama.
„Globalno“ u ovom slučaju ne znači u celom programu, već se optimizacije primenjuju na jednu po jednu funkciju. 
Ova analiza je znatno moćnija od lokalne, ali dosta komplikovanija. 
Mnoge lokalne optimizacije se  mogu se primenljivati i globalno, dok postoje i one koje se mogu primeniti samo na globalnom nivou.
Neke globalne optimizacije date su u nastavku.

\begin{itemize}
  \item \textbf{Globalna eliminacija mrtvog koda} \newline
  U ovom slučaju jedina razlika je što se informacije o tome gde je neka promenljiva živa, moraju dobiti globalnom analizom, što je značajno komplikuje u odnosu na lokalnu eliminaciju mrtvog koda.
  \item \textbf{Globalno propagiranje konstante} \newline
  Ovo je optimizacija koja svaku promenljivu za koju se zna da je konstantna menja sa tom konstantom.
  Potrebno je pratiti vrednosti koje mogu biti dodeljene promenljivoj u svakoj tački programa. 
  \item \textbf{Globalna eliminacija čestih podizraza} \newline
  Česti podizrazi mogu se eliminisati  i na globalnom nivou, poznavanjem skupa dostupnih izraza.
  Izrazi su dostupni u jednom trenutku ako su živi pri ulasku u blok, što se utvrđuje ospežnom analizom.
  \item \textbf{Optimizacija kretanje koda (eng. ~{\em Code motion})} \\
  Postoje optimizacije koje se mogu primeniti samo globalno i jedna takva je ova i ona objedinjuje sekvence koda zajedničke za jedan ili više blokova. 
  Cilj je smanjiti veličinu koda i eliminisati eventualna skupa ponovna izračunavanja. 
  Dva oblika optimizacije izdizanjem koda koje ćemo izdvojiti su :
  \begin{itemize}
    \item \textbf{Pomeranje invarijantnog koda(eng. ~{\em loop-invariant})} \\
       Podrazumeva da se računanje vrednosti neke promenljive može izdvojiti izvan petlje, ako je nezavisno od brojača petlje.
    \item \textbf{Parcijalna eliminacija suvišnosti (eng. ~{\em Partial redundancy elimination})} \\
    Za neki račun u programu kaže se da je suvišan, ako računa već poznatu vrednost. 
    Parcijalno redundantno izračunavanje  je ono čija je vrednost poznata samo u nekim delovima. 
    
  \end{itemize}
\end{itemize}

\subsubsection{Međuproceduralne optimizacije}
\label{subsubsec:međuproceduralne}
U okviru ove optimizacije jedna od najznacajnijih tehnika je \textbf{uvlačenje definicija funkcija(eng. ~{\em Inlining})}. 
Uglavnom se jednostavnije funkcije uvlače, a složenije ne. 
Premalo uvlačenja dovodi do troškova vezanih za pozive funkcija, a previše uveća veličinu koda i dovodi do neefikasnosti, pa treba uspostaviti ravnotežu.


\subsection{Optimizacije koda}
\label{subsec:optimizacija_kod}
Optimizovani međukod se u fazi generisanja koda prevodi u asemblerski tj. mašinski kod. 
U finalnom generisanju koda, bitno je donošenje pametnih odluka kako bi generisani ciljni kod bio što efikasniji. 
U ovom slučaju bitne su specifične karakteristike mašine za prozivodnju optimizovanog koda za tu konkretnu arhitekturu. 
Ralikujemo sledeće tehnike optimizacija koda:
\begin{itemize}
  \item \textbf{Optimizacija redosleda instrukcija} \\
  Razlikujemo fazu odabira instrukcija, fazu alokacije registara i fazu raspoređivanja instrukcija.
  \item \textbf{Optimizacija upotrebom keša} \\
  Upotreba keša se zasniva na prostornoj i vremenskoj lokalnosti, a cilj je da postane što bolja.
  Ako je nekoj memoriji skoro pristupano, vremenska podrazumeva da ce biti uskoro opet,a prostorna da ce i njeni susedni objekti biti uskoro korišćeni.
 
\end{itemize}
Postoje razne optimizacije koje se mogu sprovesti i na nivou izgenerisanog ciljnog koda.






\section{Prvi naslov}
\label{sec:naslov1}


Ovde pišem tekst. 



\subsection{Prvi podnaslov}
\label{subsec:podnaslov1}

Ovde pišem tekst. 


\subsection{Drugi podnaslov}
\label{subsec:podnaslov2}

Ovde pišem tekst. 



\subsection{... podnaslov}
\label{subsec:podnaslovN}

Ovde pišem tekst. 

\section{n-ti naslov}
\label{sec:naslovN}

Ovde pišem tekst. 


\subsection{... podnaslov}
\label{subsec:podnaslovK}

Ovde pišem tekst. 




\section{Zaključak}
\label{sec:zakljucak}

Ovde pišem zaključak. 



\addcontentsline{toc}{section}{Literatura}
\appendix
\bibliography{seminarski} 
\bibliographystyle{plain}

\appendix
\section{Dodatak}
Ovde pišem dodatne stvari, ukoliko za time ima potrebe.



\end{document}
